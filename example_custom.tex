%==============================================================================
% 西安交通大学学报 LaTeX 模板 - 使用示例(Overleaf兼容版)
% 本示例展示如何使用模板编写自己的论文
%==============================================================================

% 重要:在Overleaf中请将编译器设置为 XeLaTeX
% Menu -> Compiler -> XeLaTeX

\documentclass[12pt,a4paper,twoside]{article}

%------------------------------------------------------------------------------
% 宏包加载
%------------------------------------------------------------------------------
\usepackage[UTF8,fontset=none]{ctex}  % 使用none字体集,手动配置字体
\usepackage{times}
\usepackage{setspace}
\usepackage{geometry}
\usepackage{multicol}
\usepackage{graphicx}
\usepackage{amsmath}
\usepackage{amsfonts}
\usepackage{amssymb}
\usepackage{booktabs}
\usepackage{caption}
\usepackage{float}
\usepackage{array}
\usepackage{hyperref}
\usepackage{fancyhdr}
\usepackage{lastpage}

%------------------------------------------------------------------------------
% 页面设置(增加headheight以避免警告)
%------------------------------------------------------------------------------
\geometry{
    left=2.5cm,
    right=2.5cm,
    top=3cm,
    bottom=2.5cm,
    headheight=27pt  % 增加页眉高度
}

%------------------------------------------------------------------------------
% 字号定义(必须在caption设置之前定义)
%------------------------------------------------------------------------------
\newcommand{\erhao}{\fontsize{22pt}{\baselineskip}\selectfont}  % 二号
\newcommand{\xiaosihao}{\fontsize{14pt}{\baselineskip}\selectfont}  % 四号(用于标题)
\newcommand{\wuhao}{\fontsize{10.5pt}{\baselineskip}\selectfont}  % 五号
\newcommand{\xiaowuhao}{\fontsize{9pt}{\baselineskip}\selectfont}  % 小五号
\newcommand{\liuhao}{\fontsize{7.5pt}{\baselineskip}\selectfont}  % 六号

%------------------------------------------------------------------------------
% 字体设置(兼容Overleaf)
%------------------------------------------------------------------------------
% Overleaf使用Fandol字体,我们使用ctex默认设置即可
% 如果需要自定义字体,请取消下面的注释并在本地编译

% \setCJKmainfont{SimSun}      % 本地编译时使用
% \setCJKsansfont{SimHei}
% \setCJKmonofont{FangSong}

% 简化的字体命令(使用ctex的默认字体)
\newcommand{\songti}{\CJKfamily{rm}}  % 使用ctex的默认衬线字体(宋体风格)
\newcommand{\heiti}{\CJKfamily{sf}}   % 使用ctex的默认无衬线字体(黑体风格)
\newcommand{\kaishu}{\CJKfamily{rm}}  %楷体用衬线字体代替
\newcommand{\fangsong}{\CJKfamily{rm}}

%------------------------------------------------------------------------------
% 行距设置
%------------------------------------------------------------------------------
\onehalfspacing

%------------------------------------------------------------------------------
% 图表设置(现在可以使用字体命令了)
%------------------------------------------------------------------------------
\captionsetup{
    font={xiaowuhao,songti},
    labelfont={xiaowuhao,songti},
    textfont={xiaowuhao,songti},
    skip=6pt
}

%------------------------------------------------------------------------------
% 页眉页脚
%------------------------------------------------------------------------------
\pagestyle{fancy}
\fancyhf{}
\fancyhead[C]{
    \xiaowuhao\songti
    第56卷第x期\quad 2022年x月\\
    \textrm{JOURNAL OF XI'AN JIAOTONG UNIVERSITY}\quad\textrm{Vol.56 No.x}\quad\textrm{Aug. 2022}
}
\fancyfoot[C]{
    \xiaowuhao\songti
    \thepage\quad/\quad\pageref{LastPage}
}
\renewcommand{\headrulewidth}{0pt}
\renewcommand{\footrulewidth}{0pt}

%------------------------------------------------------------------------------
% 标题格式
%------------------------------------------------------------------------------
\usepackage{titlesec}
\titleformat{\section}
    {\xiaosihao\songti\centering}
    {\thesection}
    {1em}
    {}

\titleformat{\subsection}
    {\wuhao\heiti}
    {\thesubsection}
    {1em}
    {}

%------------------------------------------------------------------------------
% 文档开始
%------------------------------------------------------------------------------
\begin{document}

%===【请在此处修改期刊信息】===================================================
\begin{center}
    \vspace{-1.5cm}
    \xiaowuhao\songti
    第XX卷第X期\\  % 修改卷期
    20XX年X月\\    % 修改年月
    \textrm{JOURNAL OF XI'AN JIAOTONG UNIVERSITY}\\
    \textrm{Vol.XX No.X}\\
    \textrm{Month. 20XX}
\end{center}

\vspace{0.8cm}

%===【请在此处修改中文标题】===================================================
\begin{center}
    \erhao\heiti
    您的论文标题  % 修改标题
\end{center}

\vspace{0.4cm}

%===【请在此处修改作者信息】===================================================
\begin{center}
    \wuhao\songti
    作者一\textsuperscript{1},作者二\textsuperscript{1,2},作者三\textsuperscript{2}  % 修改作者
\end{center}

\vspace{0.3cm}

%===【请在此处修改作者单位】===================================================
\begin{center}
    \xiaowuhao\songti
    (1.第一单位名称,城市 邮编;\\  % 修改单位
    2.第二单位名称,城市 邮编)
\end{center}

\vspace{0.4cm}

%===【请在此处修改中文摘要】===================================================
\noindent\textbf{\wuhao\heiti 摘要}\wuhao\kaishu
请在这里输入您的中文摘要内容。摘要应简明扼要地说明研究目的、方法、结果和结论。通常200-400字。

\vspace{0.3cm}

\noindent\textbf{\wuhao\heiti 关键词:}\wuhao\kaishu
关键词1;关键词2;关键词3;关键词4  % 用分号分隔

\vspace{0.3cm}

\noindent\textbf{\wuhao\heiti 中图分类号:}\wuhao\kaishu
XXX  % 请填写中图分类号

\vspace{0.2cm}

\noindent\textbf{\wuhao\heiti 文献标志码:}\wuhao\kaishu
A

\vspace{0.2cm}

\noindent\xiaowuhao\songti
\textrm{DOI: 10.7652/xjtuxb20XXXXXXX}\\  % 修改DOI
\textrm{文章编号:0253-987X(20XX)XX-XXXX-XX}  % 修改文章编号

\vspace{0.6cm}

%===【请在此处修改英文标题】===================================================
\begin{center}
    \xiaosihao\heiti
    \textrm{Your English Title Goes Here}  % 修改英文标题
\end{center}

\vspace{0.4cm}

%===【请在此处修改英文作者】===================================================
\begin{center}
    \wuhao\textrm
    Author Name1\textsuperscript{1}, Author Name2\textsuperscript{1,2}  % 修改英文作者
\end{center}

\vspace{0.3cm}

%===【请在此处修改英文单位】===================================================
\begin{center}
    \liuhao\textrm
    (1. Department of First Institution, City, Postal Code, Country;\\
    2. Department of Second Institution, City, Postal Code, Country)
\end{center}

\vspace{0.4cm}

%===【请在此处修改英文摘要】===================================================
\noindent\textbf{\wuhao\heiti Abstract}\wuhao\textrm
Please enter your English abstract here. The abstract should briefly describe the research objective, methods, results, and conclusions. Usually 200-400 words.

\vspace{0.3cm}

\noindent\textbf{\wuhao\heiti Key words:}\wuhao\textrm
keyword1; keyword2; keyword3; keyword4

\vspace{0.8cm}

%===【正文开始 - 双栏排版】====================================================
\singlespacing

\begin{multicols}{2}

%---【引言】--------------------------------------------------------------------
\section{引言}

这是引言部分。在引言中,您需要:

\begin{itemize}
    \item 介绍研究背景和意义
    \item 回顾相关文献
    \item 指出存在的问题
    \item 说明本文的研究内容和贡献
\end{itemize}

文献引用示例:许多学者研究了相关问题\cite{ref1}\cite{ref2}。张三等\cite{ref3}提出了一种新方法。

%---【研究方法】----------------------------------------------------------------
\section{研究方法}

\subsection{试验装置}

介绍试验装置的组成和特点。可以使用图表来说明。

\subsection{试验方法}

详细描述试验方法和步骤。

%---【结果与讨论】--------------------------------------------------------------
\section{结果与讨论}

\subsection{试验结果}

展示和分析试验结果。

\subsection{图表使用示例}

图1是一个占位符图片。在实际使用中,请替换为您的图片:

\begin{figure}[H]
\centering
% 使用下面的命令插入实际图片
% \includegraphics[width=0.45\columnwidth]{your_figure.png}
\rule{0.45\columnwidth}{3cm}  % 这是占位符,请删除
\caption{图标题示例}
\label{fig:example}
\end{figure}

在正文中引用图片:如图\ref{fig:example}所示...

表1是一个表格示例:

\begin{table}[H]
\centering
\caption{表格标题示例}
\label{tab:example}
\begin{tabular}{lcc}
\toprule
项目 & 数值1 & 数值2 \\
\midrule
示例1 & 100 & 200 \\
示例2 & 150 & 250 \\
示例3 & 180 & 300 \\
\bottomrule
\end{tabular}
\end{table}

在正文中引用表格:如表\ref{tab:example}所示...

\subsection{数学公式示例}

行内公式示例:能量方程为 $E = mc^2$。

独立公式示例:
\[
a^2 + b^2 = c^2
\]

带编号的公式示例:
\begin{equation}
\label{eq:example}
y = ax^2 + bx + c
\end{equation}

引用公式:根据公式(\ref{eq:example})...

%---【结论】--------------------------------------------------------------------
\section{结论}

本文的主要结论如下:

(1) 结论一...

(2) 结论二...

(3) 结论三...

\end{multicols}

\vspace{0.5cm}

%===【参考文献】================================================================
\begingroup
\xiaowuhao\songti
\begin{thebibliography}{99}

% 中文文献示例
\bibitem{ref1}
\textbf{作者姓名}. 文章标题[J]. 期刊名称,年份,卷(期): 页码.

% 英文文献示例
\bibitem{ref2}
\textbf{Author Name}. Article Title[J]. Journal Name, Year, Volume(Issue): Pages.

% 请在这里添加您的参考文献
\bibitem{ref3}
\textbf{张三,李四}. 一种新的研究方法[J]. 某某学报,2020,50(3): 123-130.

\bibitem{ref4}
\textbf{Smith J, Johnson A}. An Innovative Approach[J]. Journal of Examples, 2021, 35(2): 456-465.

\end{thebibliography}
\endgroup

\end{document}
